% Copyright 2009 Dominik Wagenfuehr <dominik.wagenfuehr@deesaster.org>
% Dieses Dokument unterliegt der Creative-Commons-Lizenz
% "Namensnennung-Weitergabe unter gleichen Bedingungen 3.0 Deutschland"
% [http://creativecommons.org/licenses/by-sa/3.0/de/].

%\input{Dateien/text-voraussetzung}
%\input{Dateien/text-einleitung}

\section{Einleitung}

\begin{frame}
	\frametitle{Einleitung}

	Literatur:
	\begin{itemize}
		\item Feedback und kooperatives Lernen von Ulrike-Marie Krause\fnurl{http://www.waxmann.com/?id=20&cHash=1&buchnr=1806}
		\item E-Learning - das Drehbuch. Handbuch für Medienautoren und Projektleiter von Daniela Mair
	\end{itemize}
\end{frame}

\section{Feedbackbegriff}

\begin{frame}
	\frametitle{Feedback}
	\begin{itemize}
		\item Was ist Feedback?
		\item<2-> Feedbackquellen \small(Ilgen, Fisher, Taylor '79)
		\begin{itemize}
		    \item Andere Personen
			\item Aufgabenumfeld
			\item Das Selbst
		\end{itemize}
		\item<3-> Feedbackempfänger
	\end{itemize}
\end{frame}


\section{Funktionen}

\begin{frame}[<+->]
	\frametitle{Feedbackfunktionen}
	\begin{itemize}
		\item Kognitive
		\item Metakognitive
		\item Motivational
	\end{itemize}


	klare Trennung ist empirisch nicht möglich \small(Vroom '64)
\end{frame}

\section{Perspektiven}

\begin{frame}[<+->]
	\frametitle{Theoretische Perspektiven}
	\begin{itemize}
		\item Behavioristische Perspektive
		\begin{itemize}
			\item Positives Feedback als Verstärker
			\item Negatives Feedback als Unterdrücker
		\end{itemize}
		\item Kognitivistische Perspektive
		\begin{itemize}
			\item Feedback als Informationsquelle
		\end{itemize}
		\item Konstruktivistische Perspektive
		\begin{itemize}
		\item Feedback als Angebot für Wissenskonstruktion
		\end{itemize}
	\end{itemize}
\end{frame}

\section{Gestaltung}

\begin{frame}[<+->]
	\frametitle{Feedbackgestaltung}
	\begin{itemize}
		\item Bezugsnormorientierung
		\begin{itemize}
			\item Sachlich
			\item Individuell
			\item Sozial
		\end{itemize}
		\item Angesprochene Ebene\\
			(Aufgabe, Motivation, Selbst)
	\end{itemize}
\end{frame}

\begin{frame}
	\frametitle{Feedbackgabe}
	\begin{itemize}
		\item Bestätigend oder kritisch 
		\begin{itemize}
			\item Intention
			\item Reihenfolge der Argumente
		\end{itemize}
		\item<2-> Informierend oder kontrollierend
		\begin{itemize}
			\item Unterstützung gegen Leistungsdruck
		\end{itemize}
		\item<3-> Mündlich oder schriftlich
		\item<4-> Verständlichkeit
		\begin{itemize}
			\item ``Einfachheit'',
			\item ``Gliederung/Ordnung'',
			\item ``Kürze/Prägnanz'',
			\item ``Anregende Zusätze''
		\end{itemize}
	\end{itemize}
\end{frame}

\begin{frame}[<+->]
	\frametitle{Feedbackgabe (cont.)}
	\begin{itemize}
		\item Sofortig oder verzögert
		\begin{itemize}
			\item Verzögertes Feedback als zusätzliche Lernphase
			\item Sofortiges Feedback ist formativ
		\end{itemize}
		\item Informationsgehalt
		\begin{itemize}
			\item ``Knowledge of Results'' - richtig/falsch
			\item ``Knowledge of Correct Response'' - Musterlösung
			\item ``Try-Again-Feedback'', bzw. ``Answer-Until-Correct''
			\item ``Elaboriertes Feedback''
		\end{itemize}
	\end{itemize}
\end{frame}

\begin{frame}[<+->]
	\frametitle{Regeln zur Feedbackgabe  \small(Antons, Fengler '98)}
	Feedback sollte:
	\begin{enumerate}
		\item eher beschreiben als bewerten
		\item möglichst konkret formuliert werden anstatt allgemein
		\item eher einladen als zurechtweisen
		\item sich auf Veränderbares Verhalten beziehen und nicht auf den Charakter einer Person
		\item eher erbeten sein als aufgezwungen
		\item eher sofort gegeben werden anstatt verzögert und rekonstruierend
		\item eher klar formuliert als vage
		\item möglichst durch dritte überprüfbar sein
		\item weiteres Lernen ermöglichen
		\item auch positive Aspekte enthalten
	\end{enumerate}
\end{frame}

\begin{frame}
	\frametitle{Feedback in computergestützten Lernumgebungen}
	\begin{itemize}
		\item Adaptivität
		\begin{itemize}
			\item adaptiertes – adaptives Feedback
		\end{itemize}
	\end{itemize}
\end{frame}

