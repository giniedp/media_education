% Copyright 2009 Dominik Wagenfuehr <dominik.wagenfuehr@deesaster.org>
% Dieses Dokument unterliegt der Creative-Commons-Lizenz
% "Namensnennung-Weitergabe unter gleichen Bedingungen 3.0 Deutschland"
% [http://creativecommons.org/licenses/by-sa/3.0/de/].

% some color definition
\definecolor{orange}{rgb}{1,0.39,0.04}
\definecolor{dunkelgrau}{gray}{0.35}
\definecolor{mittelgrau}{gray}{0.85}
\definecolor{hellgrau}{gray}{0.93}
\definecolor{hellgelb}{rgb}{1.0,1.0,0.9}

\newcommand{\ListingBox}[1]
{%
    \begin{small}
        \begin{semiverbatim}
            \fcolorbox{dunkelgrau}{hellgelb}{%
                \begin{minipage}{0.95\linewidth}
                    #1
                \end{minipage}
            }
        \end{semiverbatim}
    \end{small}
}

\newcommand{\CommandBox}[1]
{%
    \begin{small}
        \begin{semiverbatim}
        \fcolorbox{dunkelgrau}{hellgrau}{%
            \begin{minipage}{0.95\linewidth}
                #1
            \end{minipage}
        }
        \end{semiverbatim}
    \end{small}
}

% URL als Fußnote
\newcommand*{\fnurl}[1]{{\footnote{{\tiny \url{#1}}}}}

% Kürzel für \textbackslash
\newcommand*{\bs}{\textbackslash}

% Definition um bei Boxen eine ganze Zeilenhöhe auszufüllen
\newcommand*{\fullh}{\vphantom{\textsuperscript{I}y}}

% Makro für Tasten
\newcommand*{\Taste}[1]{\fboxsep1pt\fbox{\fullh{}#1}}

% Makro für Tabeinrückung
\newcommand*{\tab}{\mbox{~~~~}}

% Setzt Farbe
\newcommand<>{\colors}[1]{\only<beamer>{\color#2{#1}}}

% Makro für Mischung aus uncover und color
\newcommand<>{\uncovcol}[3][]{%
    \uncover#4{%
        {%
            \ifthenelse{\equal{#1}{}}{\colors#4{#2}}{\colors#1{#2}}%
            #3%
        }%
    }%
}

% Einheitliche Beispieldatei
\newcommand*{\BspDatei}[1]{Beispiel \texttt{#1}}

% Anzeige der Beispiele als PDF
% Benutzung: \ShowExample[TRIM]{FRAMETITEL}{PDFBILD_OHNE_ENDUNG}
\newcommand{\ShowExample}[3][0 450 0 0]
{%
    \only<handout>{%
        \begin{frame}
            \frametitle{#2}
            \begin{center}
                \fbox{\includegraphics[clip,trim=#1,height=0.82\textheight]{Bilder/#3.pdf}}\\
                \textit{Abbildung: #3.pdf}
            \end{center}
        \end{frame}
    }
}
