%\vspace*{2cm}
\noindent 
Powerpoint ist eines der vielen erh�ltlichen Programme zur Erstellung von Pr�sentationsfolien. Es bietet eine Menge an n�tzlichen Funktionen die w�hrend der Gestaltung verwendet werden k�nnen. Nat�rlich kann ein Programm nicht die Bed�rfnisse aller Benutzer abdecken, so dass manch ein Benutzer die Eine oder andere Funktion vermissen wird. Diese Arbeit besch�ftigt sich damit ein Verfahren zu realisieren mit dem es m�glich ist Powerpoint um Funktionen zu erweitern. Das Verfahren sieht dazu vor dem Benutzer die M�glichkeit zu geben seine eigenen Inhalte w�hrend der Gestaltung mit Metainformationen zu belegen. Diese Informationen werden in einem Postprozess dazu verwendet um zu entscheiden wie die markierte Stelle in der Folie ver�ndert wird. Dazu wird XSL verwendet wodurch Verfahren Programmiersprachen- und Plattformunabh�ngig bleibt. Weiterhin ist zum Anwenden des Verfahrens keine kostenpflichtige Software notwendig.

