\chapter{Struktur und Aufbau des Programms}

%
%
%
\section{Navigationsstruktur}
\label{sec:navigation}
Die Abbildung der Navigationsstruktur ist im Anhang zu finden.
Grundsätzlich sind folgende Punkte gültig
\begin{itemize}
  \item Die Hauptseite ist von jeder Seite der Anwendung mit einem Klick zu erreichen.
  \item Die Modulseite ist von jeder Unterseite des Moduls mit einem Klick zu erreichen.
\end{itemize}

%
%
%
\section{Layout}
\label{sec:layout}

Das Seitenlayout definiert folgende 6 Bereiche:
\begin{description}
  \item[Kopfbereich] Hier steht der Name der Anwendung. Dieser Bereich ändert sich in der gesamten Anwendung nicht.
  \item[Inhaltsbereich] Inhalte jeder Unterseite werden hier angezeigt.
  \item[Schnellnavi] Die Schnellnavigation ermöglicht einen Schnellzugriff auf wichtige Bereiche der Anwendung. Die Navigationsknöpfe sind durch Symbole (Icons) repräsentiert. Die Navigationsleiste ist rechts ausgerichtet. Folgende Knöpfe sind immer zu sehen:
  \begin{description}
    \item[Hauptseite] \texttt{layout/main-page-icon.png} Führt immer zur Hauptseite. 
    \item[Login/Profil] \texttt{layout/login-icon.png} bzw. \texttt{layout/profile-icon.png} Führen zur Login- bzw. zur Profilseite. 
    \item[Hilfe] \texttt{layout/help-icon.png} Öffnet ein Overlay mit Hilfe zur aktuellen Seite. 
  \end{description}
  
  \item[Wo bin ich] Dreizeiliger Text mit Angabe der aktuellen Position in der Anwendung. Der Text ist immer links ausgerichtet.
  \item[Navigation] Weiterführende Navigation jeder Seite wird hier angezeigt.
  \item[Overlay] Inhalte einer Hilfeseite oder einer externen Webseite werden hier angezeigt.
	Das Overlay legt sich über die aktuelle Seite. Der Hintergrund wird dabei abgedunkelt. 
	Das Overlay lässt sich auf zwei Weisen schließen, über einen Button der sich oben rechts im Overlay befindet oder durch das Klicken außerhalb des Overlays.
\end{description}


%
%
%
\section{Hauptseite}
\label{sec:main-page}

Die Hauptseite zeigt folgende Inhalte:
\begin{description}
  \item[Textbereich] \texttt{layout/index.html} Willkommens- und Beschreibungstext der Anwendung. Der Besucher bzw. der Benutzer wird über die Ziele der Anwendung informiert und erhält Anweisungen zum weiteren Vorgehen.
  \item[Wo bin ich] Es erscheinen folgende Zeilen \emph{\\Du bist hier \\Hauptseite}
  \item[Navigation] Zeigt die fünf zuletzt verwendeten Module als einzelne Links an. Alle übrigen Module sind zusätzlich über ein Dropdownmenü erreichbar.
	
\end{description}




%
%
%
\section{Profilseite}
\label{sec:profile-page}

\begin{description}
  \item[Profilbereich] Optionen zur Accountverknüpfung (z.B. Facebook, Twitter oder ähnliche) oder Profildaten
  \item[Statistikbereich] Zeigt die ausgewählte Statistik an. Statistiken sind im Drehbuch der einzelnen Module definiert.
  \item[Wo bin ich] \emph{\\Du bist hier:\\Profilbereich\\<Benutzername>}
  \item[Navigationsbereich] Zeigt die fünf letzten Module, die eine Statistik über den Benutzer erfasst haben, als einzelne Links an. 
	Alle Module mit einer erfassten Benutzerstatistik sind zusätzlich über ein Dropdownmenü erreichbar.
\end{description}




%
%
%
\section{Modulhauptseite}
\label{sec:module-main-page}

\begin{description}
  \item[Textbereich] Beschreibt das Modul und dessen Lernziele.
  \item[Statistikbereich] 
    Falls das Modul Statistiken des Benutzers aufgezeichnet hat, so tauchen diese hier auf. Mögliche Statistiken sind:
    \begin{itemize}
      \item Anzahl falscher Antworten.
      \item Anzahl gelernter Vokabeln.
      \item etc.
    \end{itemize}
  \item[Wo bin ich] 
    \emph{\\Du bist hier:\\<Modulname>}
  \item[Navigationsbereich] Enthält folgende Navigationsknöpfe
   \begin{description}
    \item[Lernen] Verlinkt auf den Lernbereich des Moduls.
    \item[Üben] Verlinkt auf den Trainingsbereich des Moduls.
   \end{description}
\end{description}

%
%
%
\section{Modullernseite}
\label{sec:hauptseite}

\begin{description}
  \item[Seiteninhaltsbereich] Der Inhalt eines Moduls ist individuell gestaltbar und kann folgende Elemente enthalten:
  \begin{itemize}
    \item Texte
    \item Grafiken
    \item Videos
    \item Sprechertexte
    \item Externe / interne Links
  \end{itemize}
  
  \item[Wo bin ich] \emph{\\Du bist hier:\\<Modulname>\\Lernen - <Aktuelle Seite>}
  \item[Navigationsbereich] Zeigt alle Seiten des Lernbereichs in Form einer Baumstruktur an. Sollte die aktuelle Seite mit einer Trainingseinheit verknüpft sein, so wird neben dem Seitenlink ein Trainingsicon \texttt{layout/practise-icon.png} angezeigt, das auf die Trainingseinheit verlinkt.
  
\end{description}


%
%
%
\section{Modulübungsseite}
\label{sec:module-training-page}

\begin{description}
  \item[Aufgabenbereich] Art und Aussehen der Aufgaben sind in den jeweiligen Kapiteln des Drehbuchs beschrieben. Hier sollten mindestens die folgenden Elemente auftauchen:
  \begin{itemize}
    \item Aufgabenstellung
    \item Eingabefelder zur Lösung
    \item evtl. ein Aufgabe lösen Button.
  \end{itemize}
  \item[Wo bin ich] \emph{\\Du bist hier:\\<Modulname>\\Üben - <Aktuelles Training>}
  \item[Navigationsbereich] Zeigt alle Trainingseinheiten des Trainingsbereichs in einer Liste an. 
	Sollte die aktuelle Seite mit einer Lernseite verknüpft sein, so wird neben dem Link ein Lernicon \texttt{layout/learn-icon.png} angezeigt, das auf die Lernseite verlinkt.
\end{description}



\endinput 
