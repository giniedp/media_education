\chapter{ Modul - Mathematik }
\label{cha:math-module}

\section{ Hauptseite }
\label{cha:math-module-mainpage}
\begin{description}
  \item[Titel] \emph{ Hauptseite }
  \item[Textbereich] \texttt{ mathematics/index.html }
  \item[Übung] keine
  \item[Wo bin ich] \emph{\\Du bist hier:\\Mathematik\\Lernen - Hauptseite}
  \item[Navigation] Links zu den ersten Lern- und Übungsseiten
  \begin{description}
    \item[Lernen] \prettyref{cha:math-learn-page0}
    \item[Üben] \prettyref{cha:math-practise-page1}
  \end{description}
  \item[Hilfe] keine
\end{description}



\section{ Lernseiten }

\subsection{ Lernen - Hauptseite }
\label{cha:math-learn-page0}
\begin{description}
  \item[Bezeichnung] \emph{ Hauptseite }
  \item[Titel] \emph{ Hauptseite }
  \item[Textbereich] \texttt{ mathematics/learn/index.html }
  \item[Übung] keine
  \item[Wo bin ich] \emph{\\Du bist hier:\\Mathematik\\Lernen - Hauptseite}
  \item[Hilfe] keine
  \item[Überseite] keine
  \item[Unterseiten] :
  \begin{itemize}
    \item \prettyref{cha:math-learn-page1}
    \item \prettyref{cha:math-learn-page2}
    \item \prettyref{cha:math-learn-page3}
    \item \prettyref{cha:math-learn-page4}
    \item \prettyref{cha:math-learn-page5}
  \end{itemize}
\end{description}


\subsection{ Alle von 9, den letzten von 10 }
\label{cha:math-learn-page1}
\begin{description}
  \item[Bezeichnung] \emph{ Seite 1 }
  \item[Titel] \emph{ Alle von 9, den letzten von 10 }
  \item[Textbereich] \texttt{ mathematics/learn/page1.html }
  \item[Übung] \prettyref{cha:math-practise-page1}
  \item[Wo bin ich] \emph{\\Du bist hier:\\Mathematik\\Lernen - Seite 1 }
  \item[Hilfe] keine
  \item[Überseite] \prettyref{cha:math-learn-page0}
  \item[Unterseiten] keine
\end{description}



\subsection{ Vertikal und Kreuzweise }
\label{cha:math-learn-page2}
\begin{description}
  \item[Bezeichnung] \emph{ Seite 2 }
  \item[Titel] \emph{ Vertikal und Kreuzweise }
  \item[Textbereich] \texttt{ mathematics/learn/page2.html }
  \item[Übung] \prettyref{cha:math-practise-page2}
  \item[Wo bin ich] \emph{\\Du bist hier:\\Mathematik\\Lernen - Seite 2 }
  \item[Hilfe] keine
  \item[Überseite] \prettyref{cha:math-learn-page0}
  \item[Unterseiten] :
  \begin{itemize}
    \item \prettyref{cha:math-learn-page2a}
    \item \prettyref{cha:math-learn-page2b}
    \item \prettyref{cha:math-learn-page2c}
    \item \prettyref{cha:math-learn-page2d}
  \end{itemize}
\end{description}


\subsection{ Vertikal und Kreuzweise Für kleine Zahlen }
\label{cha:math-learn-page2a}
\begin{description}
  \item[Bezeichnung] \emph{ Seite 2a }
  \item[Titel] \emph{ Vertikal und Kreuzweise - für kleine Zahlen }
  \item[Textbereich] \texttt{ mathematics/learn/page2a.html }
  \item[Übung] \prettyref{cha:math-practise-page3}
  \item[Wo bin ich] \emph{\\Du bist hier:\\Mathematik\\Lernen - Seite 2a }
  \item[Hilfe] keine
  \item[Überseite] \prettyref{cha:math-learn-page2}
  \item[Unterseiten] keine
\end{description}


\subsection{ Vertikal und Kreuzweise Für Zahlen nahe 100 }
\label{cha:math-learn-page2b}
\begin{description}
  \item[Bezeichnung] \emph{ Seite 2b }
  \item[Titel] \emph{ Vertikal und Kreuzweise - für einstellige Zahlen }
  \item[Textbereich] \texttt{ mathematics/learn/page2b.html }
  \item[Übung] \prettyref{cha:math-practise-page4}
  \item[Wo bin ich] \emph{\\Du bist hier:\\Mathematik\\Lernen - Seite 2b }
  \item[Hilfe] keine
  \item[Überseite] \prettyref{cha:math-learn-page2}
  \item[Unterseiten] keine
\end{description}


\subsection{ Vertikal und Kreuzweise Für Zahlen unter 100 }
\label{cha:math-learn-page2c}
\begin{description}
  \item[Bezeichnung] \emph{ Seite 2c }
  \item[Titel] \emph{ Vertikal und Kreuzweise - für Zahlen unter 100 }
  \item[Textbereich] \texttt{ mathematics/learn/page2c.html }
  \item[Übung] \prettyref{cha:math-practise-page5}
  \item[Wo bin ich] \emph{\\Du bist hier:\\Mathematik\\Lernen - Seite 2c }
  \item[Hilfe] keine
  \item[Überseite] \prettyref{cha:math-learn-page2}
  \item[Unterseiten] keine
\end{description}


\subsection{ Vertikal und Kreuzweise Für kleine Brüche }
\label{cha:math-learn-page2d}
\begin{description}
  \item[Bezeichnung] \emph{ Seite 2d }
  \item[Titel] \emph{ Vertikal und Kreuzweise - für kleine Brüche }
  \item[Textbereich] \texttt{ mathematics/learn/page2d.html }
  \item[Übung] \prettyref{cha:math-practise-page6}
  \item[Wo bin ich] \emph{\\Du bist hier:\\Mathematik\\Lernen - Seite 2d}
  \item[Hilfe] keine
  \item[Überseite] \prettyref{cha:math-learn-page2}
  \item[Unterseiten] keine
\end{description}


\subsection{ Um 1 mehr als bei dem davor }
\label{cha:math-learn-page3}
\begin{description}
  \item[Bezeichnung] \emph{ Seite 3 }
  \item[Titel] \emph{ Um 1 mehr als bei dem davor }
  \item[Textbereich] \texttt{ mathematics/learn/page3.html }
  \item[Übung] \prettyref{cha:math-practise-page7}
  \item[Wo bin ich] \emph{\\Du bist hier:\\Mathematik\\Lernen - Seite 3 }
  \item[Hilfe] keine
  \item[Überseite] \prettyref{cha:math-learn-page0}
  \item[Unterseiten] keine
\end{description}


\subsection{ Multiplikation mit 11 }
\label{cha:math-learn-page4}
\begin{description}
  \item[Bezeichnung] \emph{ Seite 4 }
  \item[Titel] \emph{ Multiplikation mit 11 }
  \item[Textbereich] \texttt{ mathematics/learn/page4.html }
  \item[Übung] \prettyref{cha:math-practise-page8}
  \item[Wo bin ich] \emph{\\Du bist hier:\\Mathematik\\Lernen - Seite 4 }
  \item[Hilfe] keine
  \item[Überseite] \prettyref{cha:math-learn-page0}
  \item[Unterseiten] keine
\end{description}


\subsection{ Division durch 9 }
\label{cha:math-learn-page5}
\begin{description}
  \item[Bezeichnung] \emph{ Seite 5 }
  \item[Titel] \emph{ Division durch 9 }
  \item[Textbereich] \texttt{ mathematics/learn/page5.html }
  \item[Übung] \prettyref{cha:math-practise-page9}
  \item[Wo bin ich] \emph{\\Du bist hier:\\Mathematik\\Lernen - Seite 5 }
  \item[Hilfe] keine
  \item[Überseite] \prettyref{cha:math-learn-page0}
  \item[Unterseiten] keine
\end{description}




\section{ Übungsseiten }
\label{cha:math-practise}

\subsection{ Aufbau einer Übungsseite }
\label{cha:math-practise-structure}
\begin{itemize}
  \item Jede Übungsseite ist mit einer \texttt{Lernseite} verknüpft.
  \item Jede Übungsseite hat als \texttt{Hilfetext} den Inhalt der verknüpften Lernseite.
  \item Jede Übungsseite zeigt 10 zufällig generierte Aufgaben an.
  \item Eine Aufgabe besteht aus zwei Zahlen \texttt{Zahl-1}, \texttt{Zahl-2} und einem \texttt{Operator}.
  \item Jede Übung definiert den Zahlenbereich und die Menge der Operatoren.
  \item Jede Aufgabe beinhaltet ein Eingabefeld für das Ergebnis.
  \item Alle Aufgaben stehen untereinander.
\end{itemize}

Der Benutzer wählt ein Eingabefeld um seine Antwort einzutragen. 
Die Antwort wird durch die Tasten \texttt{TAB} oder \texttt{ENTER} Bestätigt. 
Die Antwort wird sofort nach Bestätigung durch das Programm geprüft. 
Neben dem Eingabefeld erscheint ein \emph{richtig} oder \emph{falsch} Icon (\texttt{layout/success-icon.png, layout/fail-icon.png}).



\subsection{ Zieh ab von 9 und den letzten von 10 }
\label{cha:math-practise-page1}
\begin{description}
  \item[Bezeichnung] \emph{ Übung 1 }
  \item[Titel] \emph{ Zieh ab von 9 und den letzten von 10 }
  \item[Textbereich] \texttt{ mathematics/practise/page1.html }
  \item[Lernseite] \prettyref{cha:math-learn-page1}
  \item[Hilfetext] \texttt{ mathematics/learn/page1.html }
  \item[Zahl-1] {10,100,1000}
  \item[Zahl-2] {[0;9], [0;99], [0;999]}
  \item[Operatoren] Subtraktion \texttt{ - }
  \item[Wo bin ich] \emph{\\Du bist hier:\\Mathematik\\Üben - Übung 1}
\end{description}



\subsection{ Vertikal und Kreuzweise 1 }
\label{cha:math-practise-page2}
\begin{description}
  \item[Bezeichnung] \emph{ Übung 2 }
  \item[Titel] \emph{ Vertikal und Kreuzweise für kleine Zahlen }
  \item[Textbereich] \texttt{ mathematics/practise/page2a.html }
  \item[Lernseite] \prettyref{cha:math-learn-page2a}
  \item[Hilfetext] \texttt{ mathematics/learn/page2a.html }
  \item[Zahl-1] [0;10]
  \item[Zahl-2] [0;10]
  \item[Operatoren] Multiplikation \texttt{ * }
  \item[Wo bin ich] \emph{\\Du bist hier:\\Mathematik\\Üben - Übung 2}
\end{description}



\subsection{ Vertikal und Kreuzweise 2 }
\label{cha:math-practise-page3}
\begin{description}
  \item[Bezeichnung] \emph{ Übung 3 }
  \item[Titel] \emph{ Vertikal und Kreuzweise für Zahlen nahe und unter 100 }
  \item[Textbereich] \texttt{ mathematics/practise/page2b.html }
  \item[Lernseite] \prettyref{cha:math-learn-page2b}
  \item[Hilfetext] \texttt{ mathematics/learn/page2b.html }
  \item[Zahl-1] [85;100]
  \item[Zahl-2] [85;100]
  \item[Operatoren] Multiplikation \texttt{ * }
  \item[Wo bin ich] \emph{\\Du bist hier:\\Mathematik\\Üben - Übung 3}
\end{description}



\subsection{ Vertikal und Kreuzweise 3 }
\label{cha:math-practise-page4}
\begin{description}
  \item[Bezeichnung] \emph{ Übung 3 }
  \item[Titel] \emph{ Vertikal und Kreuzweise für Zahlen nahe und über 100 }
  \item[Textbereich] \texttt{ mathematics/practise/page2b.html }
  \item[Lernseite] \prettyref{cha:math-learn-page2b}
  \item[Hilfetext] \texttt{ mathematics/learn/page2b.html }
  \item[Zahl-1] [100;125]
  \item[Zahl-2] [100;125]
  \item[Operatoren] Multiplikation \texttt{ * }
  \item[Wo bin ich] \emph{\\Du bist hier:\\Mathematik\\Üben - Übung 3}
\end{description}



\subsection{ Vertikal und Kreuzweise 4 }
\label{cha:math-practise-page5}
\begin{description}
  \item[Bezeichnung] \emph{ Übung 4 }
  \item[Titel] \emph{ Vertikal und Kreuzweise für Zahlen unter 100 }
  \item[Textbereich] \texttt{ mathematics/practise/page2c.html }
  \item[Lernseite] \prettyref{cha:math-learn-page2c}
  \item[Hilfetext] \texttt{ mathematics/learn/page2c.html }
  \item[Zahl-1] [0;99]
  \item[Zahl-2] [0;99]
  \item[Operatoren] Multiplikation \texttt{ * }
  \item[Wo bin ich] \emph{\\Du bist hier:\\Mathematik\\Üben - Übung 4}
\end{description}



\subsection{ Vertikal und Kreuzweise 5 }
\label{cha:math-practise-page6}
\begin{description}
  \item[Bezeichnung] \emph{ Übung 5 }
  \item[Titel] \emph{ Vertikal und Kreuzweise für kleine Brüche }
  \item[Textbereich] \texttt{ mathematics/practise/page2d.html }
  \item[Lernseite] \prettyref{cha:math-learn-page2d}
  \item[Hilfetext] \texttt{ mathematics/learn/page2d.html }
  \item[Zahl-1] [0;10]
  \item[Zahl-2] [0;10]
  \item[Operatoren] Division \texttt{ / }
  \item[Wo bin ich] \emph{\\Du bist hier:\\Mathematik\\Üben - Übung 5}
\end{description}


\subsection{ Um 1 mehr als bei dem davor }
\label{cha:math-practise-page7}
\begin{description}
  \item[Bezeichnung] \emph{ Übung 6 }
  \item[Titel] \emph{ Um 1 mehr als bei dem davor }
  \item[Textbereich] \texttt{ mathematics/practise/page3.html }
  \item[Lernseite] \prettyref{cha:math-learn-page3}
  \item[Hilfetext] \texttt{ mathematics/learn/page3.html }
  \item[Zahl-1] Zahlen die mit 5 enden
  \item[Zahl-2] Gleiche Zahl wie \texttt{ Zahl-1 }
  \item[Operatoren] Multiplikation \texttt{ * }
  \item[Wo bin ich] \emph{\\Du bist hier:\\Mathematik\\Üben - Übung 6}
\end{description}


\subsection{ Multiplikation mit 11 }
\label{cha:math-practise-page8}
\begin{description}
  \item[Bezeichnung] \emph{ Übung 7 }
  \item[Titel] \emph{ Multiplikation mit 11 }
  \item[Textbereich] \texttt{ mathematics/practise/page4.html }
  \item[Lernseite] \prettyref{cha:math-learn-page4}
  \item[Hilfetext] \texttt{ mathematics/learn/page4.html }
  \item[Zahl-1] [10;99]
  \item[Zahl-2] \texttt{ 11 }
  \item[Operatoren] Multiplikation \texttt{ * }
  \item[Wo bin ich] \emph{\\Du bist hier:\\Mathematik\\Üben - Übung 7}
\end{description}




\subsection{ Division durch 11 }
\label{cha:math-practise-page9}
\begin{description}
  \item[Bezeichnung] \emph{ Übung 8 }
  \item[Titel] \emph{ Division durch 9 }
  \item[Textbereich] \texttt{ mathematics/practise/page5.html }
  \item[Lernseite] \prettyref{cha:math-learn-page5}
  \item[Hilfetext] \texttt{ mathematics/learn/page5.html }
  \item[Zahl-1] [10;1000]
  \item[Zahl-2] \texttt{ 9 }
  \item[Operatoren] Division \texttt{ / }
  \item[Wo bin ich] \emph{\\Du bist hier:\\Mathematik\\Üben - Übung 8}
\end{description}
