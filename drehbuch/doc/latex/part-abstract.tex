\chapter{ Abstract }
\label{cha:abstract}
\emph{Das Lernprogramm \emph{Schultrainer} bietet Menschen die Möglichkeit Schulwissen
bequem zuhause oder unterwegs zu trainieren.
Es richtet sich an jeden, der sein Schulwissen auffrischen
und/oder trainieren möchte. Das trainingsorientierte Lernprogramm baut auf dem
aktuellen Lehrplan der Mittelstufe auf und wird vorzugsweise zum Üben am PC,
Laptop oder Smartphone verwendet.}
\\
\\
  Dieses Dokument Definiert die Struktur und die Inhalte des \emph{Schultrainers}, ebenso wie
zwei der Module die für die erste Version umgesetzt werden. Einige wenige Inhalte befinden sich noch in Bearbeitung und werden nachgereicht. Hervorhebungen in diesem Dokument sind wie folgt zu verstehen
\begin{itemize}
  \item \emph{Kursivschrift} Bedeutet dass dieser Text genau so in der Anwendung vorkommt. Die Position des Textes ist aus dem Context ersichtlich.
  \item \emph{Kursivschrift<mit eingeklammertem Text>} Der eingeklammerte Text ist ein Platzhalter für einen Text der durch die Anwendung ersetzt werden muss.
 \item \texttt{Schreibmaschinenschrift} Weist einen Pfad zu einem Inhaltsdokument auf. Dies sind meistens Bilddateien oder Html- Dokumente. 
 \item \textbf{Fettgedruckter Text} der in Auflisteungen vorkommt. Bezeichnet einen Bereich, Inhalt oder ienen Teil einer Seite der Anwendung.
\end{itemize}
Html- Dokumente auf die in dieser Anwendung verwiesen wird definieren weiterführende Inhalte einer Seite. Der Html Code muss nicht eins zu eins in die Anwendung übernommen werden, sondern ist eher als Richtlinie zu verstehen.